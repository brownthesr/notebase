\documentclass[12pt]{article}

% Margins
\usepackage[margin=1in]{geometry}

% AMS math packages
\usepackage{amsmath,amsthm,amssymb,amsfonts,mathtools}

% For contradiction symbol
\usepackage{marvosym}

% Line spacing
\renewcommand{\baselinestretch}{1.1}
\renewcommand{\arraystretch}{.91}

% Common math shortcuts
\newcommand{\R}{\mathbb{R}}
\newcommand{\C}{\mathbb{C}}
\newcommand{\Z}{\mathbb{Z}}
\newcommand{\Q}{\mathbb{Q}}
\newcommand{\N}{\mathbb{N}}
\newcommand{\F}{\mathbb{F}}
\newcommand{\calP}{\mathcal{P}}

\newcommand{\norm}[1]{\left\lVert#1\right\rVert}
\DeclareMathOperator{\vsspan}{span}
\DeclareMathOperator*{\argmax}{arg\,max}
\DeclareMathOperator*{\argmin}{arg\,min}

% Floor/ceiling
\DeclarePairedDelimiter\floor{\lfloor}{\rfloor}
\DeclarePairedDelimiter\ceil{\lceil}{\rceil}

% --- Theorem-style environments ---
\newtheorem{theorem}{Theorem}[section] % numbered within sections
\newtheorem{lemma}[theorem]{Lemma}     % same counter as theorems
\newenvironment{exercise}[1]{\vspace{.1in}\noindent\textbf{Exercise #1 \hspace{.05em}}}{}
\newtheorem{proposition}[theorem]{Proposition}
\newtheorem{corollary}[theorem]{Corollary}

\theoremstyle{definition}
\newtheorem{definition}[theorem]{Definition}

\theoremstyle{remark}
\newtheorem*{remark}{Remark}

%%%%%%%%%%%%%%%%%%%%%%%%%%%%%%%%%%%%%%%%%%
\begin{document}

\begin{flushright}
	\textsc{Drake Brown}  \\
	Date: Apr 14, 2025
\end{flushright}
\begin{center}
	Homework 1
\end{center}

% Exercise 5.1.10
\begin{exercise}{5.1.10}
	To prove this we just use the mean value theorem take:
	\begin{align}
		d(g(x),g(y))\leq g'(c)d(x,y)\text{ for some }c\in [a,b] \\
		\leq \alpha d(x,y)
	\end{align}
	And if $\alpha < 1$ then this converges by the banach fixed point theorem.
\end{exercise}

% Exercise 5.2.8
\begin{exercise}{5.2.8}
	To prove this we will walk through the argument
	\begin{align}
		d(y,w)=d(Tx,Tz)                                                                                                     \\
		=\sqrt{\sum\limits_{j=1}^{n}\left(\eta_j-\omega_j\right)^2}                                                         \\
		=\sqrt{\sum\limits_{j=1}^{n}(\sum_{k=1}^nc_{jk}(\xi_k-\zeta_k))^2}                                                  \\
		\leq \sum_k\sqrt{\sum\limits_{j=1}^{n}(c_{jk}(\xi_k-\zeta_k))^2}\text{ by triangle inequality since this is a norm} \\
		\leq \sum_k\sqrt{\sum\limits_{j=1}^{n}c_{jk}^2(\xi_k-\zeta_k)^2}                                                    \\
		\leq \sum_k|\xi_k-\zeta_k|\sqrt{\sum\limits_{j=1}^{n}c_{jk}^2}                                                      \\
	\end{align}
	We can now apply cauchy shwartz. set $a_k=|\xi_k-\zeta_k|,b_k=\sqrt{\sum\limits_{j=1}^{n}c_{jk}^2}$
	\begin{align}
		\leq ||x-z||_{2}\sqrt{\sum_{j,k}c_{jk}^2} \\
		\leq d(x,z)\sqrt{\sum_{j,k}c_{jk}^2}
	\end{align}
	Thus for the contraction mapping to apply we need $\sqrt{\sum_{j,k}c_{jk}^2}<1$ or equivalently $\sum_{j,k}c_{jk}^2<1$
	% TODO: Think Cauchy-shwartz
\end{exercise}

% Exercise 5.3.6
\begin{exercise}{5.3.6}
	To show this let $y$ be a limit point of $\tilde C$ by definition of limit poitn there exists $y_n\rightarrow y, y_n\in \tilde C$
	\begin{align}
		|y_n(t)-x_0|\leq c\beta
	\end{align}
	Note that since the absolute valiue function is continuous, we can take limits here and obtain"
	\begin{align}
		|y(t)-x_0|\leq c\beta
	\end{align}"
	This is because inequalities are preserved under limits (non strict at least). So $\tilde C$ contains its limit points thus it is closed
\end{exercise}


% Exercise 2.2.14
\begin{exercise}{2.2.14}
	We show that the metric:
	\begin{align}
		\tilde{d}(x,x)=0,\tilde d(x,y)+d(x,y)+1
	\end{align}
	cannot come from a norm.

	Assume that it came from a norm then it should satisfy scalar preservation:
	\begin{align}
		||ax||_{ \tilde{d}}=||ax-a0||_{ \tilde{d}} \\
		=\tilde d(ax,0)                            \\
		=d(ax,0)+1                                 \\
		=||ax||_d+1                                \\
		=|a|||x||+1
	\end{align}
	Note that we do not have the scalar preservation here. So it is not induced by a norm.
\end{exercise}
\end{document}
