\documentclass[12pt]{article}

% Margins
\usepackage[margin=1in]{geometry}

% AMS math packages
\usepackage{amsmath,amsthm,amssymb,amsfonts,mathtools}

% For contradiction symbol
\usepackage{marvosym}

% Line spacing
\renewcommand{\baselinestretch}{1.1}
\renewcommand{\arraystretch}{.91}

% Common math shortcuts
\newcommand{\R}{\mathbb{R}}
\newcommand{\C}{\mathbb{C}}
\newcommand{\Z}{\mathbb{Z}}
\newcommand{\Q}{\mathbb{Q}}
\newcommand{\N}{\mathbb{N}}
\newcommand{\F}{\mathbb{F}}
\newcommand{\calP}{\mathcal{P}}

\newcommand{\norm}[1]{\left\lVert#1\right\rVert}
\DeclareMathOperator{\vsspan}{span}
\DeclareMathOperator*{\argmax}{arg\,max}
\DeclareMathOperator*{\argmin}{arg\,min}

% Floor/ceiling
\DeclarePairedDelimiter\floor{\lfloor}{\rfloor}
\DeclarePairedDelimiter\ceil{\lceil}{\rceil}

% --- Theorem-style environments ---
\newtheorem{theorem}{Theorem}[section] % numbered within sections
\newtheorem{lemma}[theorem]{Lemma}     % same counter as theorems
\newtheorem{proposition}[theorem]{Proposition}
\newtheorem{corollary}[theorem]{Corollary}

\newenvironment{exercise}[1]{\vspace{.1in}\noindent\textbf{Exercise #1 \hspace{.05em}}}{}
\newcommand{\tr}{\text{tr}}
\theoremstyle{definition}
\newtheorem{definition}[theorem]{Definition}

\theoremstyle{remark}
\newtheorem*{remark}{Remark}

%%%%%%%%%%%%%%%%%%%%%%%%%%%%%%%%%%%%%%%%%%
\begin{document}

\begin{flushright}
	\textsc{Drake Brown}  \\
	Date: Sept 13, 2025
\end{flushright}
\begin{center}
	Explanation 1
\end{center}

So what we are trying to do is justify going from:
\begin{align}
	\sqrt{\sum\limits_{j=1}^{n}(\sum_{k=1}^nc_{jk}(\xi_k-\zeta_k))^2} \\
\end{align}
to:
\begin{align}
	\leq \sum_k\sqrt{\sum\limits_{j=1}^{n}(c_{jk}(\xi_k-\zeta_k))^2}
\end{align}
in other words why can we take the sum out of the square root? I'll justify this right here. So remember if we have a vector $x$ indexed by $x_j$ then the two norm of this vector is given by:
\begin{align}
	\norm{x}=\sqrt{\sum_{j=1}^n(x_j)^2}
\end{align}
if we set $x_j$ equal to $\sum_{k=1}^n c_{jk}(\xi_k-\zeta_k)$ then notice the whole vector x looks like:
\begin{align}
	x= % 3 x 1 BMatrix 
	\begin{bmatrix}
		\sum_{k=1}^nc_{1k}(\xi_k-\zeta_k) \\
		\vdots                            \\
		\sum_{k=1}^nc_{nk}(\xi_k-\zeta_k)
	\end{bmatrix}
\end{align}
rewriting this by factoring out the sum we get:
\begin{align}
	x=\sum_{k=1}^n\begin{bmatrix}
		              c_{1k}(\xi_k-\zeta_k) \\
		              \vdots                \\
		              c_{nk}(\xi_k-\zeta_k)
	              \end{bmatrix}
\end{align}
So x is in reality a linear combination of these c vectors. So in total we have:
\begin{align}
	\sqrt{\sum\limits_{j=1}^{n}(\sum_{k=1}^nc_{jk}(\xi_k-\zeta_k))^2} \\
	=\sqrt{\sum_{j=1}^n(x_j)^2} \text{ by definition of } x_j         \\
	=\norm{x}_2                                                       \\
	=\norm{\sum_{k=1}^n\begin{bmatrix}
			                   c_{1k}(\xi_k-\zeta_k) \\
			                   \vdots                \\
			                   c_{nk}(\xi_k-\zeta_k)
		                   \end{bmatrix}
	} \text{ by what we did in (6)}
\end{align}
From here we can now apply the triangle inequality because of the two norm here:
\begin{align}
	\leq \sum_{k=1}^n\norm{\begin{bmatrix}
			                       c_{1k}(\xi_k-\zeta_k) \\
			                       \vdots                \\
			                       c_{nk}(\xi_k-\zeta_k)
		                       \end{bmatrix}}
\end{align}
and then by definition of the two norm:
\begin{align}
	=\sum_{k=1}^n\sqrt{\sum_{j=1}^n(c_j(\xi_j-\zeta_k))^2}
\end{align}
So that is how we were able to pull out the sum. Is because there was a hidden two norm.

\end{document}
