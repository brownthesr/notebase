\documentclass[12pt]{article}

% Margins
\usepackage[margin=1in]{geometry}

% AMS math packages
\usepackage{amsmath,amsthm,amssymb,amsfonts,mathtools}

% For contradiction symbol
\usepackage{marvosym}

% Line spacing
\renewcommand{\baselinestretch}{1.1}
\renewcommand{\arraystretch}{.91}

% Common math shortcuts
\newcommand{\R}{\mathbb{R}}
\newcommand{\C}{\mathbb{C}}
\newcommand{\Z}{\mathbb{Z}}
\newcommand{\Q}{\mathbb{Q}}
\newcommand{\N}{\mathbb{N}}
\newcommand{\F}{\mathbb{F}}
\newcommand{\calP}{\mathcal{P}}

\newcommand{\norm}[1]{\left\lVert#1\right\rVert}
\DeclareMathOperator{\vsspan}{span}
\DeclareMathOperator*{\argmax}{arg\,max}
\DeclareMathOperator*{\argmin}{arg\,min}

% Floor/ceiling
\DeclarePairedDelimiter\floor{\lfloor}{\rfloor}
\DeclarePairedDelimiter\ceil{\lceil}{\rceil}

% --- Theorem-style environments ---
\newtheorem{theorem}{Theorem}[section] % numbered within sections
\newtheorem{lemma}[theorem]{Lemma}     % same counter as theorems
\newtheorem{proposition}[theorem]{Proposition}
\newtheorem{corollary}[theorem]{Corollary}

\newenvironment{exercise}[1]{\vspace{.1in}\noindent\textbf{Exercise #1 \hspace{.05em}}}{}
\newcommand{\tr}{\text{tr}}
\theoremstyle{definition}
\newtheorem{definition}[theorem]{Definition}

\theoremstyle{remark}
\newtheorem*{remark}{Remark}

%%%%%%%%%%%%%%%%%%%%%%%%%%%%%%%%%%%%%%%%%%
\begin{document}

\begin{flushright}
	\textsc{Drake Brown}  \\
	Date: Apr 14, 2025
\end{flushright}
\begin{center}
	Homework 1
\end{center}

%%%%%%%%%%%%%%%%%%%%%%%%%%%%%%%%%%%%%%%%%%
% Examples of the new environments

\begin{theorem}[Pythagoras]
	For a right triangle with sides $a,b$ and hypotenuse $c$,
	\begin{equation}
		a^2 + b^2 = c^2.
	\end{equation}
\end{theorem}

\begin{proof}
	We compute:
	\begin{align*}
		(a+b)^2 & = a^2 + 2ab + b^2, \\
		(a-b)^2 & = a^2 - 2ab + b^2.
	\end{align*}
	Subtracting and rearranging gives the desired result.
\end{proof}

\begin{lemma}
	For $n \in \N$,
	\[
		\sum_{k=1}^n k = \frac{n(n+1)}{2}.
	\]
\end{lemma}

\begin{remark}
	You can replace the word “Theorem” with “Lemma”, “Proposition”, “Definition”, etc.
\end{remark}

%%%%%%%%%%%%%%%%%%%%%%%%%%%%%%%%%%%%%%%%%%
\end{document}
