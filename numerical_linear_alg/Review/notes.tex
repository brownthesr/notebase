\documentclass[12pt]{article}
% This package simply sets the margins to be 1 inch.
\usepackage[margin=1in]{geometry}
\usepackage{enumerate}
\usepackage{amsmath,amsthm,amssymb}
\usepackage{amsfonts}
\usepackage{mathtools}
\usepackage{marvosym}% This adds the contradiction command (\Lightning)
% These packages include nice commands from AMS-LaTeX

% Make the space between lines slightly more
% generous than normal single spacing, but compensate
% so that the spacing between rows of matrices still
% looks normal.  Note that 1.1=1/.9090909...
\renewcommand{\baselinestretch}{1.1}
\renewcommand{\arraystretch}{.91}

% Define an environment for exercises.
\newenvironment{exercise}[1]{\vspace{.1in}\noindent\textbf{Exercise #1 \hspace{.05em}}}{}

% define shortcut commands for commonly used symbols
\newcommand{\R}{\mathbb{R}}
\newcommand{\dt}{\Delta t}
\newcommand{\C}{\mathbb{C}}
\newcommand{\Z}{\mathbb{Z}}
\newcommand{\Q}{\mathbb{Q}}
\newcommand{\N}{\mathbb{N}}
\newcommand{\F}{\mathbb{F}}
\newcommand{\calP}{\mathcal{P}}
\newcommand{\nR}{\mathcal{R}}
\newcommand{\E}{\mathbb{E}}
\newcommand{\tr}{\text{tr}}
\newcommand{\xk}{x_k}
\newcommand{\yk}{y_k}
\newtheorem*{remark}{Remark}
\newcommand{\norm}[1]{\left\lVert#1\right\rVert}
\DeclareMathOperator{\vsspan}{span}
\DeclareMathOperator*{\argmax}{arg\,max}
\DeclareMathOperator*{\argmin}{arg\,min}
\begin{document}
\DeclarePairedDelimiter\floor{\lfloor}{\rfloor}
\DeclarePairedDelimiter\ceil{\lceil}{\rceil}

%%%%%%%%%%%%%%%%%%%%%%%%%%%%%%%%%%%%%%%%%%

\begin{flushright}
	\textsc{Drake Brown}  \\
	Date Apr 14, 2025
\end{flushright}
\begin{center}
	Review for Exams
\end{center}
There are a couple of different ways that one could address this. The easiest is to just do add another gridpoint and solve

To solve for stability of PDEs, first one can use MOL and disecretize the equation. From here just verify that all of the eigenvalues lie within the stability region. Alternatively one could use Neumann analysis and the lax-rictemeyer stability theorem.

Assume that $\rho(A)<1$ Then we can decompose our matrix into its jordan canonical form:
\begin{align*}
	A=QSQ^T
\end{align*}

So to prove the thing with matrix splits. Assume that a matrix A is irreducibly diagonally dominant. Note then that at least one row in
\begin{align*}
	D^{-1}(-L-U)
\end{align*}
will sum to something less than one. thus we know that if we take some eigenvector:
\begin{align*}
	(D^{-1}(-L-U)v)_i<v_i
\end{align*}
So either that entry is zero or the eigenvalue is less than one.


If it is less than one then we have proven it.

By the definition of irreducibility there are no nontrivial coordinate invariant subspaces. Assume that the entry of the eigenvector is zero. That means that if we just keep iterating by multiplying on the left by $D^{-1}(-L-U)$ the at most all of the other entries are invariant coordinate subspace. Contradicting the fact that its irreducible
\section{Mink theorem}
Let A be a full ranke matrix. Let B be some rank r matrix. Let $A_r$ be the rank r approximation to A. Note that if we take one of the first $r+1$ right singular vectors of eA tn because B is only rank r we know that at least one of them lies in the null space of b. Call it index j. We will take the supremum over all unit vectors x
\begin{align*}
	\norm{A-B}  \sup \norm{(A-B)x}\geq \norm{(A-B)v_j}=\norm{Av} = \sigma_j\norm{u_j}\geq \sigma_{r+1} \\
	\norm{A-A_{r}}=\norm{\Sigma-\Sigma_{r}}=\sigma_{r+1}
\end{align*}

Now that is just for the two norm. For the spectral norm notice that:
\begin{align*}
	\norm{A-B}_F=\norm{\Sigma - Z}                                                                       \\
	=\sum_i\sum_j(\Sigma_{ij}-Z_{ij})^2                                                                  \\
	=\sum_{ij}\Sigma_{ij}^2 -\Sigma_{ij}\bar Z_{ij} - Z_{ij}\Sigma_{ij} + |Z_{ij}|^2                     \\
	=\sum_{i} \Sigma_{ii} - \sum_{i}(\Sigma_{ij}\bar Z_{ij} + Z_{ij}\Sigma_{ij}) + \sum_{ij}|Z_{ij}|^2   \\
	\geq\sum_{i} \Sigma_{ii} - \sum_{i}(\Sigma_{ii}\bar Z_{ii} + Z_{ii}\Sigma_{ii}) + \sum_{i}|Z_{ii}|^2 \\
	=\sum_{i} \Sigma_{ii} - \sum_{i}(\Sigma_{ii}\text{real} (Z_{ii})) + \sum_{i}|Z_{ii}|^2               \\
	\geq \sum_{i} \Sigma_{ii} - \sum_{i}(\Sigma_{ii}|Z_{ii}|)  + \sum_{i}|Z_{ii}|^2                      \\
	\geq \sum_{i} \Sigma_{ii} - \Sigma_{ii}|Z_{ii}| + |Z_{ii}|^2                                         \\
	=\sum_{i}(\Sigma_{ii}-Z_{ii})^2                                                                      \\
\end{align*}
And this is minimized exactly when the Z is equal to the all of the largest singular values. thus:
\begin{align*}
	Z=\Sigma_r       \\
	U^TBV = \Sigma_r \\
	B = U\Sigma_rV^T=A_r
\end{align*}
And we have proven it
\section{Uniqueness}
\subsection{Cholesky}

Assume there were two with the regular normalizations:
jj
\begin{align*}
	L^TL=U^TU                \\
	I = (L^T)^{-1}U^TUL^{-1} \\
	I = (UL^{-1})^{T}UL^{-1} \\
\end{align*}

From this we knwo that $UL^{-1}$ is unitary. However note that it is also upper triangular. That means that it is actually diagonal and that the diagonal entries must be the roots of unity. From the fact that we know that the diagonal entries of U are real and positive, we also know that the diagonal entries of $L^{-1}$ are real and positive. Thus they must be one. so:

\begin{align*}
	UL^{-1}=I \\
	\implies U=L
\end{align*}
\subsection{LU}
This proceeds in a very similar way to the previous proof. let:
\begin{align*}
	LU=\tilde{L} \tilde{U}
\end{align*}
By the same logic as befoQe we know that $U^{-1}\tilde{U}$ must be equal to the idenity.
\subsection(QR)
\begin{align}
	kL
\end{align}
\begin{align*}
	QR=\tilde{Q}\tilde{R} \\
	Q^T\tilde{Q}=R^{-1}\tilde{R}
\end{align*}
Thus we know that $Q^T\tilde{Q}$ is unitaQy and diagonal. By the same logic as cholesky it must be equal to the identity
\section{ODE stuff}
\begin{align*}
	A_0+A_1+A_2=2        \\
	A_1+2A_2=2           \\
	A_1+4a_2=\frac{8}{3} \\
\end{align*}
Then:
\begin{align}
	2A_2=\frac{8}{3}-2 \\
	A_2=\frac{4}{3}-1  \\
	A_2=\frac{1}{3}    \\
	A_1=\frac{4}{3}    \\
	A_0=\frac{1}{3}
\end{align}
The stability polynomial is :
\begin{align*}
	x^2-3x+2=(x-1)(x-)
\end{align*}
The solution to this is $c_0+c_12^x$
\section{2024 Test}
\end{document}

