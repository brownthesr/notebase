\documentclass[12pt]{article}
% This package simply sets the margins to be 1 inch.
\usepackage[margin=1in]{geometry}
\usepackage{enumerate}
\usepackage{amsmath,amsthm,amssymb}
\usepackage{amsfonts}
\usepackage{mathtools}
\usepackage{marvosym}% This adds the contradiction command (\Lightning)
% These packages include nice commands from AMS-LaTeX

% Make the space between lines slightly more
% generous than normal single spacing, but compensate
% so that the spacing between rows of matrices still
% looks normal.  Note that 1.1=1/.9090909...
\renewcommand{\baselinestretch}{1.1}
\renewcommand{\arraystretch}{.91}

% Define an environment for exercises.
\newenvironment{exercise}[1]{\vspace{.1in}\noindent\textbf{Exercise #1 \hspace{.05em}}}{}

% define shortcut commands for commonly used symbols
\newcommand{\R}{\mathbb{R}}
\newcommand{\dt}{\Delta t}
\newcommand{\C}{\mathbb{C}}
\newcommand{\Z}{\mathbb{Z}}
\newcommand{\Q}{\mathbb{Q}}
\newcommand{\N}{\mathbb{N}}
\newcommand{\F}{\mathbb{F}}
\newcommand{\calP}{\mathcal{P}}
\newcommand{\nR}{\mathcal{R}}
\newcommand{\E}{\mathbb{E}}
\newcommand{\tr}{\text{tr}}
\newcommand{\xk}{x_k}
\newcommand{\yk}{y_k}
\newtheorem*{remark}{Remark}
\newcommand{\norm}[1]{\left\lVert#1\right\rVert}
\DeclareMathOperator{\vsspan}{span}
\DeclareMathOperator*{\argmax}{arg\,max}
\DeclareMathOperator*{\argmin}{arg\,min}
\begin{document}
\DeclarePairedDelimiter\floor{\lfloor}{\rfloor}
\DeclarePairedDelimiter\ceil{\lceil}{\rceil}

%%%%%%%%%%%%%%%%%%%%%%%%%%%%%%%%%%%%%%%%%%

\begin{flushright}
	\textsc{Drake Brown}  \\
	Date Apr 14, 2025
\end{flushright}
\begin{center}
	Review for Exams
\end{center}
\section{Numerical Linear Algebra }

\subsection{Convergence of Matrix Splits}
Common Problems
\begin{itemize}
	\item prove that a given matrix split Converges
	\item prove the gershgorin disc theorem
	\item Matrix norms (Prove 1s or infinity norm of a matrix)
\end{itemize}

Common Methods/tools:
\begin{itemize}
	\item Gershgorin Disc theorem
	\item irreducible diagonally dominant matrices are invertible
	\item Perron Frobenius theorem
	\item write out the cholesky decomposition
	\item write out definition (even the stupid supremums or limits)
	\item gershgorin disc theorem
	\item irreducible diagonally dominant matrices are invertible
	\item perron frobenius theorem
	\item write out the cholesky decomposition
	\item gershgorin disc theorem
	\item irreducible diagonally dominant matrices are invertible
	\item perron frobenius theorem
\end{itemize}

\subsection{Decompositions}
Common Problems
\begin{itemize}
	\item Perform Householder Decompositions
	\item Prove Multiplication by unitary matrix is backwards stable
	\item growth factors of LU with pivoting
	\item Positive definite matrices have nonzero diagonals
	\item prove existence of SVD
\end{itemize}

Common tools:
\begin{itemize}
	\item Write out exact algorithm
	\item Unitary matrices are preserved under 2 and F norms.
	\item Definition of Backwards stability
\end{itemize}

\subsection{Error Analysis}
Common Problems
\begin{itemize}
	\item Prove something is stable
	\item Prove something is backwards stable
	\item Find the relative error (forward error)
	\item find the absolute error
	\item find the (relative) condition number of a problem
\end{itemize}

Common Tools

\section{Numerical Methods for ODEs/PDEs}

\subsection{Convergence for IVPs}
Common Problems:
\begin{itemize}
	\item Prove that given multi-step method is consistent
	\item Prove the given multi-step method is stable
	\item Give an example of why being unstable could ruin things
	\item prove that something is Lax-richetmeyer stable
\end{itemize}

Common methods for solving such problems
\begin{itemize}
	\item Taylor expansion
	\item $p(s)-z\sigma(s)$
	\item Neumann Analysis
	\item equivalence theorems
	\item Consistency + Lax Richtemeyer stablity $\implies$ Convergence
	\item Dhalquist Stability theorem
	\item Prove zero stability
\end{itemize}

\subsection{BVP}
Common Problems
\begin{itemize}
	\item Give the extrapolation method
	\item Describe method for solving nonlinear equations
	\item Give error formula
	\item Write out what the matrices might look like
\end{itemize}

Tools for solving problems
\begin{itemize}
	\item Taylor Series
\end{itemize}
\\
\subsection{Advection Equation stuff}
Common Problems
\begin{itemize}
	\item Give the Lax-freidericks, Wendroff, or upwind method
	\item Determine what the particular method is equivalent to.
	\item Talk about why the upwind method is the way it is.
\end{itemize}

\end{document}
